\section{Assignment \#2 on  a Mac/Linux Box}

1.  Put the {\tt gmtl} folder in your {\tt /usr/local/include} directory.  

\

2.  Somehow, some of the {\tt \#include} statements repeat.  I don't know if that's a problem, but take the repetition out.  

\begin{verbatim}
#include <stdlib.h>
#include <stdio.h>


#include <gmtl\gmtl.h>
#include <stdlib.h>
#include <stdio.h>
\end{verbatim}

3.  Line 9:  Change 

\begin{verbatim}#include <GL/glew.h>\end{verbatim} 

to  

\begin{verbatim}#include <GL\glew.h>\end{verbatim}

4.  In line 24, the {\tt fopen\_s} will give you trouble.  

\begin{verbatim}
	fopen_s(&fptr, file, "rb");	
\end{verbatim}


All of the {\tt *\_s} stuff is new to the C11 standard, which is not widely supported.  Since it only appears once, you could rewrite that line as:

\begin{verbatim}
	fptr = fopen ( file, "rb");
\end{verbatim}

If some function like this occurs a lot, you could redefine it near the top of your file:

\begin{verbatim}
#ifdef __APPLE__
    #define fopen_s(pFile,filename,mode) ((*(pFile))=fopen((filename),(mode)))==NULL
#endif
\end{verbatim}


\

5.  As we did in the very first assignment, uncomment/add these lines at line 58:

\begin{verbatim}
    glfwWindowHint(GLFW_CONTEXT_VERSION_MAJOR, 3);
    glfwWindowHint(GLFW_CONTEXT_VERSION_MINOR, 3);
    glfwWindowHint(GLFW_OPENGL_FORWARD_COMPAT, GL_TRUE);
    glfwWindowHint(GLFW_OPENGL_PROFILE, GLFW_OPENGL_CORE_PROFILE);
\end{verbatim}

\

6.  As Lou mentioned in her forum post, in the {\tt OpenGL\_Example.frag} file, the {\tt gl\_FragColor} should be a dummy variable, but actually means something.  In lines 3 and 7, change both of them to the same thing.  I called them ``Fred.''

\

7.  I compiled with these flags and libraries.  

\begin{verbatim}
$ g++ C00412257_Assignment_02.cpp -framework OpenGL -lGLEW -lGLFW -w
\end{verbatim}

