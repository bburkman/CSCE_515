%%%%%%%%%%%%%%%%
\section{Math Review:  Algebra}
%%%%%%%%%%%%%%%%%%

A {\bf group} is a set $S$ with an operation (``+'')  such that:

\qquad The set S is {\it closed} under the operation, meaning that if $a,b \in S$, then $a+b \in S$.

\qquad The operation is associative, meaning that if $a,b,c \in S$, then $a+(b+c) = (a+b)+c$.

\qquad The set $S$ contains a unit element (``0''), such that $a+0 = 0+a = a \ \forall \ a \in S.$

\qquad For every element $a \in S$, there is an inverse element, $-a$, such that $a+(-a) = -a + a = 0$.

If the operation is commutative, meaning $a+b = b+a \ \forall \ a, b, \in S$, then $S$ is called an {\bf Abelian group.}

\

A {\bf ring} is a set $R$ with two operations, $+$ and $\cdot$, such that:

\qquad The set $R$ is an Abelian group.  

\qquad The set $R$ is closed under multiplication.

\qquad Multiplication is associative.  

\qquad The distributive laws hold:

\qquad \qquad $a \cdot (b+c) = a \cdot b + a \cdot c$

\qquad \qquad $(b+c) \cdot a = b \cdot a + c \cdot a$

A ring with a multiplicative identity (``1'') is called a {\bf ring with unit}.

If the ring has the property that if $a\cdot b = 0$ then $a=0$ or $b=0$, it is called a {\bf domain}.

If multiplication is commutative, then the ring is called a {\bf commutative ring}.

If a domain is commutative, then it is called an {\bf integral domain}.

\

A {\bf field} is a commutative ring with unit element (``1'') such that every nonzero element has an inverse.  

\

Quaternions are not a field because multiplication of quaternions is not commutative; however, every nonzero quaternion has a multiplicative inverse, so they are called a {\bf division ring}.

